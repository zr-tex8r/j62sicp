\documentclass[uplatex,dvipdfmx,a4paper,papersize]{jsbook}
\usepackage[T1]{fontenc}
\usepackage{lmodern}
\usepackage{hyperref,pxjahyper}
\hypersetup{%
  colorlinks=false
  bookmarks=true,
  bookmarksnumbered=true,
  pdftitle={非公式SICP [完璧にSICPをマスターして
    これ以上読む必要が全くない人向けの] 日本語版},
  pdfauthor={誰か},
  pdfsubject={これは単なるジョークです}
}
\usepackage{pgffor}
\setlength{\textwidth}{\fullwidth}
\newcommand\chapterText{アレ。}
\begin{document}
%======================================= ふろんとまたー
\frontmatter
% 表題ページ
\begin{titlepage}
  \newcommand\fSize[1]{\fontsize{#1}{#1}\selectfont}
  \centering
  \vspace*{\stretch{1}}
  {\fSize{40}\gtfamily 非 公 式}\par
  \vspace{80pt}
  {\fSize{100}\bfseries SICP}\par
  \vspace{80pt}
  {\huge 完璧にSICPをマスターして\\
  これ以上読む必要が\\
  全くない人向けの\par}
  \vspace{30pt}
  {\fSize{50}\gtfamily 日本語版}\par
  \vspace*{\stretch{2}}
\end{titlepage}

% 目次
\tableofcontents

%---------
\chapter*{序文}
\addcontentsline{toc}{chapter}{序文}
「えっ!? お前情報系なのに、これまでに1度もSICPを
翻訳したことがないの!?」\cite{internet}
と言われて、ついカッとなってやった。
完成はしてない。

%======================================= めいんまたー
\mainmatter
% 第1章~第5章
\foreach \n in {1,...,5} {
  \chapter{SICPの第\n 章}
  \chapterText
}

%======================================= ばっくまたー
\backmatter
% 参考文献
\begin{thebibliography}{9}
\bibitem{internet} 情報 インターネット
\end{thebibliography}

%======================================= それではまたー
\end{document}
